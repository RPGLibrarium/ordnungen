%%%
% Document Class
\documentclass[11pt,a4paper,twoside,openany,article]{memoir}
% openright: each chapter will start on a recto page.
%
%%%

\input{templates/jura}

%%%
% Set information for title page

\title{ Satzung \\ RPG Librarium Aachen e.V.}
\date{16. September 2019}
%%%

\begin{document}
  \maketitle
  \sloppy
%%%
% Hier könnte man bei größeren Ordnungen Kapitel auf machen
% \chapter{Allgemeines}
% %%
\begin{para}{Name, Sitz, Geschäftsjahr}
  \label{p:name}
  \item Der Verein führt den Namen „RPG Librarium Aachen“ (kurz „Librarium“) und soll in das Vereinsregister eingetragen werden; durch die Eintragung führt er den Zusatz „e.~V.“.
  \item Der Verein hat seinen Sitz in Aachen.
  \item Das Geschäftsjahr des Vereins ist das Kalenderjahr.
\end{para}

\begin{para}{Zweck des Vereins}
  \label{p:zweck}
  \item Der Zweck des Vereins ist die Förderung von Kunst und Kultur und der Volksbildung, indem Mitgliedern und Freunden des Vereins das Rollen- und Simulationsspiel ermöglicht und dieses gefördert wird.
  \item Die Förderung der Kunst und Kultur wird insbesondere verwirklicht durch
  \begin{subpara}
    \item die Organisation von Gruppen für Laien- und Improvisationstheater ohne Zuschauer (Rollenspiel),
    \item die Schaffung einer Sammlung von Literatur zum Thema Rollen- und Simulationsspiele,
    \item der Vernetzung von Menschen in Aachen und Umgebung, die Interesse an Rollen- und Simulationsspielen zeigen.
  \end{subpara}
  \item Die Förderung der Volksbildung wird insbesondere verwirklicht
  \begin{subpara}
    \item auf dem Bereich der Friedensbildung durch die angeleitete Simulation von erdachten Situationen mit persönlichen und gesellschaftlichen Konflikten, mit dem Ziel die gewaltfreie Konfliktlösung zu erlernen,
 Werte zu erlernen,
    \item durch die Durchführung von Veranstaltungen mit fortbildendem Charakter zum Thema Rollen- und Simulationsspiele.
  \end{subpara}
\end{para}

\begin{para}{Gemeinnützigkeit}
  \label{p:gemeinnützigkeit}
  \item Der Verein verfolgt ausschließlich und unmittelbar gemeinnützige Zwecke im Sinne des Abschnitts „Steuerbegünstigte Zwecke“ der Abgabenordnung.
  \item Der Verein ist selbstlos tätig; er verfolgt nicht in erster Linie eigen\-wirtschaftliche Zwecke.
  \item Mittel des Vereins dürfen nur für die satzungsmäßigen Zwecke verwendet werden. Die Mitglieder erhalten keine Gewinnanteile und in ihrer Eigenschaft als Mitglieder auch keine sonstigen Zuwendungen aus den Mitteln des Vereins. Es darf keine Person durch Ausgaben, die dem Zweck des Vereins fremd sind, oder durch unverhältnismäßig hohe Vergütungen begünstigt werden.
  \item Bei Auflösung des Vereins oder bei Wegfall steuerbegünstigter Zwecke fällt das Vermögen an den „Verein der Alumni der Fachschaft Mathematik/Physik/Informatik an der RWTH Aachen e.V.“, der es unmittelbar und ausschließlich für gemeinnützige, mildtätige oder kirchliche Zwecke zu verwenden hat.
  \item Alle Inhaber von Vereinsämtern sind ehrenamtlich tätig.
\end{para}

\begin{para}{Mitgliedschaften}
  \label{p:mitgliedschaft}
  \item Es gibt drei Arten von Mitgliedern:
    \begin{subpara}
    \item ordentliche Mitglieder,
    \item Fördermitglieder und
    \item Ehrenmitglieder.
    \end{subpara}
\end{para}

\begin{para}{Erwerb der Mitgliedschaft}
  \label{p:mitgliedschafterwerb}
  \item Jede natürliche Person kann ordentliches Mitglied werden.
  \item Jede natürliche oder juristische Person kann Fördermitglied werden.
  \item Zum Erlangen der Mitgliedschaft muss jede natürliche Person das zwölfte Lebensjahr erreicht haben. Bei Minderjährigen ist der Beitrittsantrag auch von einem gesetzlichen Vertreter zu unterschreiben.
  \item Über den schriftlichen Antrag entscheidet der Vorstand. Der Antrag soll den Namen, eine gültige E-Mail-Adresse und die Anschrift des Antragstellers ent\-halten.
  \item Gegen den ablehnenden Bescheid des Vorstands, der mit Gründen zu versehen ist, kann binnen einer Frist von einem Monat nach Zugang des Bescheids schriftlich Einspruch bei der Mitgliederversammlung eingelegt werden.
  \item Jede natürliche Person kann von der Mitgliederversammlung zum Ehrenmitglied ernannt werden.
\end{para}

\begin{para}{Beendigung der Mitgliedschaft}
  \label{p:mitgliedschaftende}
  \item Die Mitgliedschaft endet
  \begin{subpara}
  \item durch Tod oder – bei juristischen Personen – durch Auflösung.
  \item durch Austritt.
  Der Austritt erfolgt durch schriftliche Erklärung gegenüber einem Mitglied des Vorstands mit einer Frist von einem Monat zum Ende eines Halbjahres.
  \item durch Ausschluss.
  Der Ausschluss ist durch Beschluss des Vorstands möglich, wenn das Mitglied gegen die Vereinsinteressen gröblich verstoßen hat. Der schriftlich begründete Beschluss ist dem Mitglied bekannt zu machen.
  Gegen diesen Beschluss kann binnen einer Frist von einem Monat nach Zugang der Ausschlusserklärung schriftlich Einspruch bei der Mitgliederversammlung eingelegt werden. Bis zur Eröffnung der Mitgliederversammlung ruhen die Rechte des Mitglieds.
  \item durch Streichung aus der Mitgliederliste.
  Die Streichung aus der Mitgliederliste kann durch den Vorstand erfolgen, wenn das Mitglied mit seinem Mitgliedsbeitrag länger als zwei Zahlungstermine in Verzug ist und trotz Mahnung den Rückstand nicht innerhalb von zwei Wochen ausgeglichen hat. In der Mahnung muss das Mitglied auf die bevorstehende Streichung aus der Mitgliederliste hingewiesen werden.
  \end{subpara}
\end{para}

\begin{para}{Mitgliedsbeiträge}
  \label{p:mitgliedsbeiträge}
  \item Von den Mitgliedern werden Beiträge erhoben. Die Höhe des Mitgliedsbeitrages und dessen Fälligkeit werden von der Mitgliederversammlung bestimmt.
  \item Ehrenmitglieder sind von der Beitragspflicht befreit.
\end{para}

\begin{para}{Materialverleih}
  \label{p:materialverleih}
  \item Ordentliche Mitglieder und Ehrenmitglieder des Vereins können sich Material und Bücher vom Verein ausleihen. Das nähere Verfahren bestimmt eine von der Mitgliederversammlung gegebene Verleihordnung.
\end{para}

\begin{para}{Organe des Vereins}
\label{p:organe}
\item Organe des Vereins sind
  \begin{subpara}
  \item der Vorstand,
  \item die Mitgliederversammlung.
  \end{subpara}
\end{para}

\begin{para}{Vorstand}
  \label{p:vs}
  \item Der Vorstand des Vereins besteht mindestens aus dem Vorsitzenden, dem stellvertretenden Vorsitzenden und dem Schatzmeister.
  Er kann um bis zu zwei Beisitzer ergänzt werden.
  \item Vorstand im Sinne des § 26 BGB sind der Vorsitzende und der stellvertretende Vorsitzende. Sie müssen volljährig sein. Jeder von ihnen vertritt den Verein alleine.
\end{para}

\begin{para}{Zuständigkeit des Vorstands}
  \label{p:vszuständigkeit}
  \item Der Vorstand führt die Alltagsgeschäfte, soweit diese nicht durch die Satzung einem anderen Vereinsorgan zugewiesen sind. Er hat unter anderem folgende Aufgaben:
  \begin{subpara}
    \item Anschaffung von Regel- und Quellbänden, sowie sonstigem Material,
    \item Vorbereitung der Mitgliederversammlungen und Aufstellung der Tagesordnungen,
    \item Einberufung der Mitgliederversammlung,
    \item Ausführung der Beschlüsse der Mitgliederversammlung,
    \item Aufstellung eines Haushaltsplans für jedes Geschäftsjahr und Buchführung,
    \item Beschlussfassung über Aufnahme und Ausschluss von Mitgliedern,
    \item Führung der Mitgliederliste.
    Der Vorstand ist dazu angehalten, sich bei wichtigen Angelegenheiten die Meinung der Mitglieder einzuholen.
  \end{subpara}
  \item Rechtsgeschäfte sowohl im Innen- als auch im Außenverhältnis mit einem Geschäftswert über 500 € sind für den Verein nur verbindlich, wenn hierzu die Zustimmung der Mitgliederversammlung erteilt wurde.
\end{para}

\begin{para}{Wahl des Vorstands}
  \label{p:vswahl}
  \item Der Vorstand wird von der Mitgliederversammlung für die Dauer von einem Jahr, gerechnet von der Wahl an, gewählt; er bleibt jedoch bis zur Wahl des neuen Vorstands im Amt.
  \item Eine Wiederwahl ist möglich.
  \item Jedes Amt ist einzeln zu wählen.
  \item Wählbar sind nur ordentliche Mitglieder und Ehrenmitglieder.
  \item Die Vereinigung mehrerer Vorstandsämter in einer Person ist nicht zulässig.
  \item Scheidet der Vorstandsvorsitzende, sein Stellvertreter oder der Schatzmeister während der Amtsperiode aus, so wählt der restliche Vorstand ein Ersatzmitglied für die übrige Amtsdauer des Ausgeschiedenen.
\end{para}

\begin{para}{Arbeitsweise des Vorstands}
  \label{p:vsarbeitsweise}
  \item Der Vorstand fasst seine Beschlüsse im Allgemeinen in Vorstandssitzungen, die vom Vorsitzenden, bei dessen Verhinderung vom stellvertretenden Vorsitzenden, spätestens am Vortag in Textform einberufen werden.
  \item Der Vorstand ist beschlussfähig, wenn mehr als die Hälfte der Vorstandsmitglieder anwesend sind. Die Beschlussfassung erfolgt mit absoluter Mehrheit.
  \item Die Beschlüsse des Vorstands sind zu Protokoll zu bringen, welches von zwei Anwesenden zu unterschreiben ist. Das Protokoll soll außerdem Ort und Zeit der Vorstandssitzung sowie die Namen der Teilnehmer enthalten.
  \item Ein Vorstandsbeschluss kann auch ohne Vorstandssitzung einstimmig und auf schriftlichem Wege gefasst werden.
\end{para}

\begin{para}{Mitgliederversammlung}
  \label{p:mv}
  \item Die Mitgliederversammlung besteht aus der Gesamtheit der Mitglieder.
  \item Die Mitgliederversammlung ist insbesondere zuständig für
  \begin{subpara}
    \item die Entgegennahme des Berichts der Kassenprüfer und Wahl des Kassenprüfer,
    \item den Beschluss zur Entlastung des Vorstands und Wahl sowie Abberufung des Vorstands,
    \item die Festsetzung der Beitragsordnung, sowie von Gebühren und Umlagen und deren Fälligkeit und Erhebungsform,
    \item die Genehmigung des Haushaltsplans,
    \item die Beratung von Einsprüchen gegen Beschlüsse des Vorstands und deren Annahme,
    \item den Beschluss von Satzungs- und Zweckänderungen,
    \item die Ernennung von Ehrenmitgliedern,
    \item den Beschluss über die Auflösung des Vereins.
  \end{subpara}
\end{para}

\begin{para}{Einberufung der Mitgliederversammlung}
  \label{p:mveinberufung}
  \item Mindestens einmal im Geschäftsjahr soll eine ordentliche Mitgliederversammlung stattfinden. Sie wird vom Vorstand unter Einhaltung einer Frist von zwei Wochen per Textform unter Angabe der geplanten Tagesordnung einberufen.
  \item Das Einladungsschreiben gilt dem Mitglied als zugegangen, wenn es an die letzte vom Mitglied dem Verein bekannt gegebene E-Mail-Adresse gerichtet ist.
  \item Die Tagesordnung kann durch Mehrheitsbeschluss der Mitgliederversammlung in der Sitzung geändert oder auf Antrag eines einzelnen Mitglieds vor Beginn der Mitgliederversammlung ergänzt werden.
  \item Anträge zur Satzungsänderung sind mit einer Frist von einer Woche vor der Mitgliederversammlung an den Vorstand zu stellen.
\end{para}

\begin{para}{Beschlussfassung der Mitgliederversammlung}
  \label{p:mvbeschlussfassung}
  \item Die Mitgliederversammlung gibt sich eine Geschäftsordnung, in der ein Versammlungsleiter bestimmt wird.
  \item Die Art der Abstimmung bestimmt der Versammlungsleiter. Die Abstimmung muss geheim durchgeführt werden, wenn ein anwesendes Mitglied dies beantragt.
  \item Jedes ordentliche Mitglied und jedes Ehrenmitglied hat eine Stimme.
  \item Zur Ausübung des Stimmrechts kann ein anderes Mitglied schriftlich bevoll\-mächtigt werden. Die Bevollmächtigung ist für jede Mitgliederversammlung gesondert zu erteilen. Ein Mitglied darf jedoch nicht mehr als drei fremde Stimmen vertreten.
  \item Die Mitgliederversammlung ist beschlussfähig, wenn mindestens ein Drittel der stimmberechtigten Mitglieder anwesend sind.
  Bei Beschlussunfähigkeit ist der Vorstand verpflichtet, innerhalb von vier Wochen eine zweite Mitglieder\-ver\-sammlung mit der gleichen Tagesordnung unter Berücksichtigung der oben genannten Frist einzuberufen; diese ist ohne Rücksicht auf die Zahl der erschienenen Mitglieder beschlussfähig. Hierauf ist in der Einladung hinzuweisen.
  \item Die Mitgliederversammlung fasst Beschlüsse im Allgemeinen mit einfacher Mehrheit der abgegebenen gültigen Stimmen; Stimmenthaltungen bleiben daher außer Betracht.
  \item Zur Änderung der Satzung ist eine Mehrheit von drei Vierteln der abgegebenen gültigen Stimmen erforderlich.
  \item Zur Auflösung des Vereins ist eine Mehrheit von drei Vierteln der abgegebenen gültigen Stimmen erforderlich.
  \item Für Wahlen gilt Folgendes: Hat im ersten Wahlgang kein Kandidat die absolute Mehrheit aller Stimmen erreicht, findet eine Stichwahl zwischen den Kandidaten statt, welche die beiden höchsten Stimmzahlen erreicht haben. Gewählt ist dann derjenige, der die meisten Stimmen auf sich vereint. Bei Gleichstand entscheidet das Los.
  \item Über die Beschlüsse der Mitgliederversammlung muss ein Protokoll geführt werden, das vom jeweiligen Versammlungsleiter, dem Protokollführer und einem weiteren Mitglied zu unterzeichnen ist.
  Es soll außerdem die folgenden Informationen enthalten:
  \begin{subpara}
    \item Ort und Zeit der Versammlung,
    \item die Person des Versammlungsleiters und des Protokollführers,
    \item die Zahl der erschienenen Mitglieder,
    \item die einzelnen Abstimmungsergebnisse und die Art der Abstimmung.
  \end{subpara}
  Bei Satzungs- und Zweckänderungen soll der genaue Wortlaut angegeben werden.
\end{para}

\begin{para}{Außerordentliche Mitgliederversammlungen}
  \label{p:mvaußerordentlich}
  \item Der Vorstand kann jederzeit eine außerordentliche Mitgliederversammlung einberufen. Diese muss einberufen werden, wenn
  \begin{subpara}
    \item das Interesse des Vereins es erfordert oder wenn
    \item diese von einem Fünftel aller Mitglieder schriftlich unter Angabe des Zwecks und der Gründe verlangt wird.
    Für die außerordentliche Mitgliederversammlung gelten die \parasref{p:mv}, \ref{p:mveinberufung} und \ref{p:mvbeschlussfassung} entsprechend.
  \end{subpara}
\end{para}

\begin{para}{Kassenprüfer}
  \label{p:kassenprüfer}
  \item Mindestens zwei Kassenprüfer werden für die Dauer der Amtszeit des Vorstandes von der Mitgliederversammlung gewählt. Sie dürfen nicht dem Vorstand angehören.
  \item Die Kassenprüfer können umfassend Einsicht in alle Verwaltungs- und Finanzunterlagen des Vereins nehmen. Sie prüfen im Auftrag und im Interesse der Mitgliederversammlung, insbesondere im Hinblick auf
  \begin{subpara}
    \item die Ordentlichkeit der Verwaltungs- und Finanzunterlagen,
    \item die Wirtschaftlichkeit und Sparsamkeit,
    \item die Einhaltung der Beschlüsse der Mitgliederversammlung,
    \item die Satzungsgemäßheit.
  \end{subpara}
  \item Die Vereinsorgane unterstützen die Kassenprüfer bei der Erfüllung ihres Auftrags.
  \item Die Kassenprüfer berichten der Mitgliederversammlung jährlich über Art, Umfang und Ergebnisse ihrer Prüfung und geben ein Votum zur Beschlussfassung über die Entlastung des Vorstands ab.
\end{para}

\begin{para}{Auflösung des Vereins, vertretungsberechtigte Liquidatoren}
  \label{p:auflösung}
  \item Die Auflösung des Vereins kann nur in einer Mitgliederversammlung mit der im \pararef{p:mvbeschlussfassung} festgelegten Stimmenmehrheit beschlossen werden.
  \item Sofern die Mitgliederversammlung nichts anderes beschließt, sind der Vorsitzende und der stellvertretende Vorsitzende gemeinsam vertretungs\-be\-rech\-tigte Liquidatoren.
  Die vorstehenden Vorschriften gelten entsprechend für den Fall, dass der Verein aus einem anderen Grund aufgelöst wird oder seine Rechtsfähigkeit verliert.
\end{para}

\begin{changes}
\item Die vorstehende Satzung wurde in der Gründungsversammlung vom 27. September 2014 errichtet.
\item Diese Satzung wurde am 31.01.2015 verändert.
\item Diese Satzung wurde am 11.10.2015 verändert.
\item Diese Satzung wurde am 23.04.2017 verändert.
\item Diese Satzung wurde am 16.08.2017 verändert.
\item Diese Satzung wurde am 02.03.2019 verändert.
\item Diese Satzung wurde am 16.09.2019 beschlossen.
\end{changes}

\end{document}
