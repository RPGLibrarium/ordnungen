%%%
% Document Class
\documentclass[11pt,a4paper,twoside,openany,article]{memoir}
% openright: each chapter will start on a recto page.
%
%%%

\input{templates/jura}

%%%
% Set information for title page

\title{Ordnung der RPG Librarium Föderation}
\date{TODO}
%%%

\begin{document}
  \maketitle
  \sloppy
%%%
% Hier könnte man bei größeren Ordnungen Kapitel auf machen
% \chapter{Allgemeines}
% %%
\begin{para}{Preambel}
\label{p:preambel}
\item Die RPG Librarium Föderation ist ein Zusammenschluss von Vereinen, die sich zur Förderung des Rollen- und Simulationsspiels berufen fühlen und gemeinsam unter einem Namen auf dem Gebiet des Planeten Erde tätig werden. Im Rahmen des Verbandes vernetzen Sie sich mit dem Ziel allen Menschen  das Rollen- und Simulationsspiel zu ermöglichen.
\end{para}

\begin{para}{Name, Sitz, Geschäftsjahr}
\label{p:name}
\item Der Verband führt den Namen \enquote{RPG Librarium Föderation}.
\item Föderation hat Ihren Sitz am Wohnort des Chronisten.
\item Das Geschäftsjahr der Föderation ist das Kalenderjahr.
\end{para}

\begin{para}{Organe}
\label{p:organe}
\item Organe des Verbands sind
    \begin{subpara}
    \item der Chronist, 
    \item die Delegiertenkonferenz.
    \end{subpara}
\end{para}

\begin{para}{Mitgliedschaft}
\label{p:mitgliedschaft}
\item Jeder Verein oder auf Dauer angelegte Zusammenschluss von Menschen kann Mitglied werden.
\item Der Beitrittsantrag ist schriftlich mit einer Frist von 2 Wochen vor der Delegiertenkonferenz an den Chronisten zu richten und ist von den jeweiligen Vertretungsberechtigten zu unterschreiben.
\item Über die Aufnahme entscheidet die Delegiertenkonferenz.
\item Die Mitgliedschaft endet 
    \begin{subpara}
    \item durch Auflösung des Mitglieds oder
    \item durch schriftliche Erklärung gegenüber des Chronisten mit einer Frist von 30 Tagen zur nächsten Großen Konjunktion von Jupiter und Saturn.
    \end{subpara}
\end{para}

\begin{para}{Rechte der Mitglieder}
\label{p:rechte}
\item Das Mitglied hat das Recht den Namen \enquote{RPG Librarium} zu führen. Der Name ist durch eine Orts- oder Regionsangabe zu ergänzen und soll sich von den Namen anderer Mitglieder außreichend unterscheiden.
\item Das Mitglied hat das Recht sich mit dem gemeinsamen Corporate Design nach außen darzustellen.
\end{para}

\begin{para}{Pflichten der Mitglieder}
\label{p:pflichten}
\item Die Mitglieder sind selbstlos tätig und verfolgen nicht in erster Linie eigenwirtschaftliche Zwecke.
\item Die Mitglieder investieren nach eigener Möglichkeit in die Anschaffung und Bereitstellung von Literatur und Material zum Thema Rollen- und Simulationsspiele.
\item Die Mitglieder fördern in ihrem Gebiet die Vernetzung von Menschen, die Interesse an Rollen- und Simulationsspiel zeigen. 
\item Die Mitglieder berichten jährlich dem Chronisten von Ihren Tätigkeiten und Engagement bei der Erfüllung des gemeinsamen Zwecks.
\end{para}

\begin{para}{Chronist}
\label{p:chronist}
\item Der Chronist führt die Geschäfte der Föderation und spricht für ihn. Er hat unter anderem folgende Aufgaben:
    \begin{subpara}
    \item Führung der Mitgliederliste,
    \item Vorbereitung und Einberufung der Delegiertenkonferenz,
    \item Ausführung der Beschlüsse der Delegiertenkonferenz,     
    \item Koordinierung der und Kommunikation mit den Mitgliedern.
    \item Archivierung der Berichte der Mitglieder. 
    \end{subpara}
\item Der Chronist ist ehrenamtlich tätig.
\item Die Delegiertenkonferenz wählt aus ihrer Mitte den Chronisten für die Dauer von drei Jahren. Eine Wiederwahl ist nicht möglich.
\item Das Amt des Chronisten soll unter den Mitgliedern rotieren.
\item Verstirbt der Chronist während seiner Amtszeit, bestellt das älteste Mitglied der Föderation einen Barden, der die Aufgaben des Chronisten bis zur nächsten Delegiertenkonferenz kommissarisch übernimmt.
\end{para}

\begin{para}{Delegiertenkonferenz}
\label{p:delegiertenkonferenz}
\item Die Delegiertenkonferenz ist das höchste beschlussfassende Organ der Föderation.
\item Der Delegiertenkonferenz gehören mit Sitz und Stimme 
    \begin{subpara}
    \item jeweils zwei Delegierte pro Mitglied und
    \item der Chronist an.
    \end{subpara}
\item Die Delegiertenkonferenz tritt mindestens einmal jährlich zusammen. Sie wird vom Chronisten mit einer Frist von 30 Tagen einberufen. Eine Deligiertenkonferenz muss außerordentlich einberufen werden, wenn mindestens ein Drittel der Mitglieder der Föderation dieses schriftlich beim Chronisten beantragen.
\item Die Delegiertenkonferenz ist insbesondere zuständig für
    \begin{subpara}
    \item die Wahl des Chronisten,
    \item die Aufnahme  von Mitgliedern,
    \item den Beschluss von Ordnungsänderungen,
    \item die Behandlung von Grundsatz- und Konzeptionsfragen,
    \item die Verabschiedung von Resolutionen und Anträgen,
    \item den Beschluss über die Auflösung der Föderation.
    \end{subpara}
\end{para}

\begin{para}{Beschlussfassung der Delegiertenkonferenz}
\label{p:beschlussfassung}
\item Die Delegiertenkonferenz ist beschlussfähig wenn sie ordentlich einberufen wurde.
\item Die Delegiertenkonferenz wird vom Chronisten geleitet. Er bestimmt die Art der Abstimmung. Die Abstimmung muss geheim durchgeführt werden, wenn ein Delegierter dies beantragt.
\item Die Delegiertenkonferenz fasst Beschlüsse im Allgemeinen mit einfacher Mehrheit der abgegebenen gültigen Stimmen; Stimmenthaltungen bleiben daher außer Betracht.
\item Zur Änderung der Ordnung ist eine Mehrheit von drei Vierteln der abgegeben gültigen Stimmen erforderlich.
\item Für Wahlen gilt Folgendes: Hat im ersten Wahlgang kein Kandidat die einfache Mehrheit der abgegebenen Stimmen erreicht, findet eine Stichwahl zwischen den Kandidaten statt, welche die beiden höchsten Stimmzahlen erreicht haben. Gewählt ist dann derjenige, der die meisten Stimmen auf sich vereint. Bei Gleichstand entscheidet der Würfel.
\item Über die Beschlüsse der Delegiertenkonferenz führt der Chronist ein Protokoll, das allen Delegierten und den Mitgliedern der Föderation zeitnah zuzustellen ist.
Bei Abwesenheit des Chronisten übernimmt der ältesten anwesenden Deligierte seine Aufgaben.
\end{para}

\begin{para}{Auflösung}
\label{p:auflösung}
\item Auf Antrag des Chronisten oder eines Mitglieds kann die Föderation von der Delegiertenkonferenz einstimmig aufgelöst werden. Auf die Auflösung ist in der Einladung hinzuweisen.
\end{para}

\begin{changes}
\item Die vorstehende Ordnung wurde nicht beschlossen.\end{changes}


\end{document}

